\documentclass[12pt]{article}


\usepackage[portuguese]{babel}
\usepackage[utf8]{inputenc}
\usepackage{listings}
\usepackage{tikz}
\usepackage{booktabs}
\usepackage{float}
\usepackage{indentfirst}
\usepackage{titlesec}
\usepackage{amsmath}
\usepackage{subcaption}
\usepackage{hyperref}
\usepackage{textcomp}
\usepackage{tcolorbox}

\title{
    \Huge Processamento de Linguagens\\
    \normalsize \textbf{Trabalho Prático 1}
}

\date{\today}

\setlength{\parindent}{2em}

\begin{document}

\begin{titlepage}

    \maketitle
    \begin{center}\large

        \begin{tabular}{ll}
            \textbf{Grupo} nr. & 36
            \\\hline
            a83899 & André Morais
            \\
            a85954 & Luís Ribeiro
            \\
            a84783 & Pedro Rodrigues
        \end{tabular}

    \end{center}

    \vfill

    \begin{figure}[h]
        \centering
        \includegraphics[height=3cm]{images/UM.png}
    \end{figure}

    \begin{center}
        \large Mestrado Integrado em Engenharia Informática\\
        Universidade do Minho
    \end{center}

\end{titlepage}

\tableofcontents{}

\newpage

\section{Introdução}

\vspace{1cm}

\subsection{Contexto}

\vspace{0.5cm}

Este relatório foi produzido em conformidade com a UC de \textbf{Processamento de Linguagens}, correspondente ao segundo semestre do terceiro ano do Mestrado Integrado em Engenharia Informática da Universidade do Minho.\\

\subsection{Problema}

\vspace{0.5cm}

Para vários projetos de desenvolvimento, é habitual haver soluções envolvendo vários ficheiros e várias pastas, como \texttt{Makefile}, \texttt{README} ou uma paste de exemplos.

Pretende-se, então, criar um programa \textbf{mkfromtemplate}, capaz de aceitar um nome de projeto e um ficheiro de descrição (template) e que crie os ficheiros e as pastas do peojeto, bem como escreva o conteúdo dos ficheiros pretendido.

O template deverá ser constituído por:
\begin{itemize}
  \item metadados (author e email)
  \item tree (estrutura de diretorias e ficheiros a criar)
  \item template de cada ficheiro
\end{itemize}

Um exemplo do template pode ser encontrado em \textbf{FAZER REFERÊNCIA}.


\newpage

\section{Análise e Especificação}



\newpage

\section{Concepção/desenho da Resolução}

\newpage

\section{Codificação e Testes}

\newpage

\section{Conclusão}

\newpage

\appendix
\section{Código do Programa}

\end{document}
